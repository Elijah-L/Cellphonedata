% Options for packages loaded elsewhere
\PassOptionsToPackage{unicode}{hyperref}
\PassOptionsToPackage{hyphens}{url}
%
\documentclass[
]{article}
\usepackage{lmodern}
\usepackage{amssymb,amsmath}
\usepackage{ifxetex,ifluatex}
\ifnum 0\ifxetex 1\fi\ifluatex 1\fi=0 % if pdftex
  \usepackage[T1]{fontenc}
  \usepackage[utf8]{inputenc}
  \usepackage{textcomp} % provide euro and other symbols
\else % if luatex or xetex
  \usepackage{unicode-math}
  \defaultfontfeatures{Scale=MatchLowercase}
  \defaultfontfeatures[\rmfamily]{Ligatures=TeX,Scale=1}
\fi
% Use upquote if available, for straight quotes in verbatim environments
\IfFileExists{upquote.sty}{\usepackage{upquote}}{}
\IfFileExists{microtype.sty}{% use microtype if available
  \usepackage[]{microtype}
  \UseMicrotypeSet[protrusion]{basicmath} % disable protrusion for tt fonts
}{}
\makeatletter
\@ifundefined{KOMAClassName}{% if non-KOMA class
  \IfFileExists{parskip.sty}{%
    \usepackage{parskip}
  }{% else
    \setlength{\parindent}{0pt}
    \setlength{\parskip}{6pt plus 2pt minus 1pt}}
}{% if KOMA class
  \KOMAoptions{parskip=half}}
\makeatother
\usepackage{xcolor}
\IfFileExists{xurl.sty}{\usepackage{xurl}}{} % add URL line breaks if available
\IfFileExists{bookmark.sty}{\usepackage{bookmark}}{\usepackage{hyperref}}
\hypersetup{
  pdftitle={TP\_EX3},
  pdfauthor={vhinyg},
  hidelinks,
  pdfcreator={LaTeX via pandoc}}
\urlstyle{same} % disable monospaced font for URLs
\usepackage[margin=1in]{geometry}
\usepackage{color}
\usepackage{fancyvrb}
\newcommand{\VerbBar}{|}
\newcommand{\VERB}{\Verb[commandchars=\\\{\}]}
\DefineVerbatimEnvironment{Highlighting}{Verbatim}{commandchars=\\\{\}}
% Add ',fontsize=\small' for more characters per line
\usepackage{framed}
\definecolor{shadecolor}{RGB}{248,248,248}
\newenvironment{Shaded}{\begin{snugshade}}{\end{snugshade}}
\newcommand{\AlertTok}[1]{\textcolor[rgb]{0.94,0.16,0.16}{#1}}
\newcommand{\AnnotationTok}[1]{\textcolor[rgb]{0.56,0.35,0.01}{\textbf{\textit{#1}}}}
\newcommand{\AttributeTok}[1]{\textcolor[rgb]{0.77,0.63,0.00}{#1}}
\newcommand{\BaseNTok}[1]{\textcolor[rgb]{0.00,0.00,0.81}{#1}}
\newcommand{\BuiltInTok}[1]{#1}
\newcommand{\CharTok}[1]{\textcolor[rgb]{0.31,0.60,0.02}{#1}}
\newcommand{\CommentTok}[1]{\textcolor[rgb]{0.56,0.35,0.01}{\textit{#1}}}
\newcommand{\CommentVarTok}[1]{\textcolor[rgb]{0.56,0.35,0.01}{\textbf{\textit{#1}}}}
\newcommand{\ConstantTok}[1]{\textcolor[rgb]{0.00,0.00,0.00}{#1}}
\newcommand{\ControlFlowTok}[1]{\textcolor[rgb]{0.13,0.29,0.53}{\textbf{#1}}}
\newcommand{\DataTypeTok}[1]{\textcolor[rgb]{0.13,0.29,0.53}{#1}}
\newcommand{\DecValTok}[1]{\textcolor[rgb]{0.00,0.00,0.81}{#1}}
\newcommand{\DocumentationTok}[1]{\textcolor[rgb]{0.56,0.35,0.01}{\textbf{\textit{#1}}}}
\newcommand{\ErrorTok}[1]{\textcolor[rgb]{0.64,0.00,0.00}{\textbf{#1}}}
\newcommand{\ExtensionTok}[1]{#1}
\newcommand{\FloatTok}[1]{\textcolor[rgb]{0.00,0.00,0.81}{#1}}
\newcommand{\FunctionTok}[1]{\textcolor[rgb]{0.00,0.00,0.00}{#1}}
\newcommand{\ImportTok}[1]{#1}
\newcommand{\InformationTok}[1]{\textcolor[rgb]{0.56,0.35,0.01}{\textbf{\textit{#1}}}}
\newcommand{\KeywordTok}[1]{\textcolor[rgb]{0.13,0.29,0.53}{\textbf{#1}}}
\newcommand{\NormalTok}[1]{#1}
\newcommand{\OperatorTok}[1]{\textcolor[rgb]{0.81,0.36,0.00}{\textbf{#1}}}
\newcommand{\OtherTok}[1]{\textcolor[rgb]{0.56,0.35,0.01}{#1}}
\newcommand{\PreprocessorTok}[1]{\textcolor[rgb]{0.56,0.35,0.01}{\textit{#1}}}
\newcommand{\RegionMarkerTok}[1]{#1}
\newcommand{\SpecialCharTok}[1]{\textcolor[rgb]{0.00,0.00,0.00}{#1}}
\newcommand{\SpecialStringTok}[1]{\textcolor[rgb]{0.31,0.60,0.02}{#1}}
\newcommand{\StringTok}[1]{\textcolor[rgb]{0.31,0.60,0.02}{#1}}
\newcommand{\VariableTok}[1]{\textcolor[rgb]{0.00,0.00,0.00}{#1}}
\newcommand{\VerbatimStringTok}[1]{\textcolor[rgb]{0.31,0.60,0.02}{#1}}
\newcommand{\WarningTok}[1]{\textcolor[rgb]{0.56,0.35,0.01}{\textbf{\textit{#1}}}}
\usepackage{graphicx,grffile}
\makeatletter
\def\maxwidth{\ifdim\Gin@nat@width>\linewidth\linewidth\else\Gin@nat@width\fi}
\def\maxheight{\ifdim\Gin@nat@height>\textheight\textheight\else\Gin@nat@height\fi}
\makeatother
% Scale images if necessary, so that they will not overflow the page
% margins by default, and it is still possible to overwrite the defaults
% using explicit options in \includegraphics[width, height, ...]{}
\setkeys{Gin}{width=\maxwidth,height=\maxheight,keepaspectratio}
% Set default figure placement to htbp
\makeatletter
\def\fps@figure{htbp}
\makeatother
\setlength{\emergencystretch}{3em} % prevent overfull lines
\providecommand{\tightlist}{%
  \setlength{\itemsep}{0pt}\setlength{\parskip}{0pt}}
\setcounter{secnumdepth}{-\maxdimen} % remove section numbering

\title{TP\_EX3}
\author{vhinyg}
\date{28/09/2020}

\begin{document}
\maketitle

\begin{Shaded}
\begin{Highlighting}[]
\NormalTok{tab =}\StringTok{ }\KeywordTok{read.table}\NormalTok{(}\StringTok{"cellphonedata.txt"}\NormalTok{, }\DataTypeTok{header =}\NormalTok{ T,}\DataTypeTok{sep =} \StringTok{";"}\NormalTok{,}\DataTypeTok{dec =} \StringTok{","}\NormalTok{)}

\NormalTok{X =}\StringTok{ }\KeywordTok{cbind}\NormalTok{(}\KeywordTok{rep}\NormalTok{(}\DecValTok{1}\NormalTok{,}\KeywordTok{length}\NormalTok{(tab[,}\DecValTok{1}\NormalTok{])}\OperatorTok{-}\DecValTok{2}\NormalTok{),tab[}\DecValTok{1}\OperatorTok{:}\NormalTok{(}\KeywordTok{length}\NormalTok{(tab[,}\DecValTok{1}\NormalTok{])}\OperatorTok{-}\DecValTok{2}\NormalTok{),}\DecValTok{1}\NormalTok{]) }
\end{Highlighting}
\end{Shaded}

Affichages des données

\begin{Shaded}
\begin{Highlighting}[]
\NormalTok{Ytest =}\StringTok{ }\NormalTok{tab[}\DecValTok{1}\OperatorTok{:}\NormalTok{(}\KeywordTok{length}\NormalTok{(tab[,}\DecValTok{1}\NormalTok{])}\OperatorTok{-}\DecValTok{2}\NormalTok{),}\KeywordTok{c}\NormalTok{(}\DecValTok{2}\NormalTok{)]}
\NormalTok{Yfut =}\StringTok{ }\NormalTok{tab[(}\KeywordTok{length}\NormalTok{(tab[,}\DecValTok{1}\NormalTok{])}\OperatorTok{-}\DecValTok{1}\NormalTok{)}\OperatorTok{:}\NormalTok{(}\KeywordTok{length}\NormalTok{(tab[,}\DecValTok{1}\NormalTok{])),}\KeywordTok{c}\NormalTok{(}\DecValTok{2}\NormalTok{)]}
\NormalTok{Xtest =}\StringTok{ }\NormalTok{tab[}\DecValTok{1}\OperatorTok{:}\NormalTok{(}\KeywordTok{length}\NormalTok{(tab[,}\DecValTok{1}\NormalTok{])}\OperatorTok{-}\DecValTok{2}\NormalTok{),}\KeywordTok{c}\NormalTok{(}\DecValTok{1}\NormalTok{)]}
\NormalTok{Xfut =}\StringTok{ }\NormalTok{Yfut =}\StringTok{ }\NormalTok{tab[(}\KeywordTok{length}\NormalTok{(tab[,}\DecValTok{1}\NormalTok{])}\OperatorTok{-}\DecValTok{1}\NormalTok{)}\OperatorTok{:}\NormalTok{(}\KeywordTok{length}\NormalTok{(tab[,}\DecValTok{1}\NormalTok{])),}\KeywordTok{c}\NormalTok{(}\DecValTok{1}\NormalTok{)]}
\NormalTok{X1 =}\StringTok{ }\KeywordTok{cbind}\NormalTok{(}\KeywordTok{rep}\NormalTok{(}\DecValTok{1}\NormalTok{,}\KeywordTok{length}\NormalTok{(tab[,}\DecValTok{1}\NormalTok{])}\OperatorTok{-}\DecValTok{2}\NormalTok{),Xtest)}
\KeywordTok{plot}\NormalTok{(Xtest,Ytest,}\DataTypeTok{col =} \StringTok{'blue4'}\NormalTok{,}\DataTypeTok{xlab =} \StringTok{"Year"}\NormalTok{,}\DataTypeTok{ylab =} \StringTok{"Number of sale"}\NormalTok{,}\DataTypeTok{panel.first =} \KeywordTok{grid}\NormalTok{(}\DecValTok{10}\NormalTok{))}
\end{Highlighting}
\end{Shaded}

\includegraphics{exo4_files/figure-latex/unnamed-chunk-2-1.pdf}

The trend of the curve let us think that a linear model doesn't fit with
data. The trend let us think about a \(atan\) or an \(tanh\) function
between the model \(\mathbf{Y}\) and the variable \(\mathbf{X}\) However
we are going to give a chance to the linear model then see his
limitations on predictions

\begin{Shaded}
\begin{Highlighting}[]
\KeywordTok{cor}\NormalTok{(tab)}
\end{Highlighting}
\end{Shaded}

\begin{verbatim}
##               year  EndUsers
## year     1.0000000 0.9509108
## EndUsers 0.9509108 1.0000000
\end{verbatim}

However, because \(\rho_{xy} = 0.95\) we are going to give a chance to
the linear model then see his limitations on predictions. First it's to
notice that this cross correlations term is close to 1 because most of
data are in the middle of the distribution

\mathbf{The linear model}

Estimation de \(\^{\beta}\) and \(\^{Y}\)

\begin{Shaded}
\begin{Highlighting}[]
\NormalTok{tab1 =}\StringTok{ }\NormalTok{tab[}\DecValTok{1}\OperatorTok{:}\NormalTok{(}\KeywordTok{length}\NormalTok{(tab[,}\DecValTok{1}\NormalTok{])}\OperatorTok{-}\DecValTok{2}\NormalTok{),]}
\NormalTok{(tab1)}
\end{Highlighting}
\end{Shaded}

\begin{verbatim}
##    year EndUsers
## 1  2007   122.32
## 2  2008   139.29
## 3  2009   172.38
## 4  2010   296.65
## 5  2011   472.00
## 6  2012   680.11
## 7  2013   969.72
## 8  2014  1244.74
## 9  2015  1423.90
## 10 2016  1495.96
## 11 2017  1536.27
## 12 2018  1556.27
## 13 2019  1524.84
\end{verbatim}

\begin{Shaded}
\begin{Highlighting}[]
\NormalTok{modreg =}\StringTok{ }\KeywordTok{lm}\NormalTok{(EndUsers }\OperatorTok{~}\StringTok{ }\NormalTok{. , }\DataTypeTok{data =}\NormalTok{ tab1)}
\KeywordTok{summary}\NormalTok{(modreg)}
\end{Highlighting}
\end{Shaded}

\begin{verbatim}
## 
## Call:
## lm(formula = EndUsers ~ ., data = tab1)
## 
## Residuals:
##     Min      1Q  Median      3Q     Max 
## -260.95 -126.01  -13.31  118.19  232.00 
## 
## Coefficients:
##               Estimate Std. Error t value Pr(>|t|)    
## (Intercept) -297979.58   23611.08  -12.62 6.92e-08 ***
## year            148.47      11.73   12.66 6.71e-08 ***
## ---
## Signif. codes:  0 '***' 0.001 '**' 0.01 '*' 0.05 '.' 0.1 ' ' 1
## 
## Residual standard error: 158.2 on 11 degrees of freedom
## Multiple R-squared:  0.9358, Adjusted R-squared:  0.9299 
## F-statistic: 160.2 on 1 and 11 DF,  p-value: 6.706e-08
\end{verbatim}

\begin{Shaded}
\begin{Highlighting}[]
\NormalTok{betah =}\StringTok{ }\NormalTok{modreg}\OperatorTok{$}\NormalTok{coefficients}
\NormalTok{betah }\CommentTok{#coefficient beta hat}
\end{Highlighting}
\end{Shaded}

\begin{verbatim}
##  (Intercept)         year 
## -297979.5765     148.4722
\end{verbatim}

\begin{Shaded}
\begin{Highlighting}[]
\NormalTok{Yhat =}\StringTok{ }\NormalTok{X1}\OperatorTok\NormalTok{betah}
\end{Highlighting}
\end{Shaded}

plot of the data estimated trend

\begin{Shaded}
\begin{Highlighting}[]
\NormalTok{\{}\KeywordTok{plot}\NormalTok{(Xtest,Yhat,}\DataTypeTok{xlab =} \StringTok{"Year"}\NormalTok{,}\DataTypeTok{ylab =} \StringTok{"Number of sale"}\NormalTok{,}\DataTypeTok{panel.first =} \KeywordTok{grid}\NormalTok{(}\DecValTok{10}\NormalTok{)) }
\KeywordTok{abline}\NormalTok{(betah[}\DecValTok{1}\NormalTok{],betah[}\DecValTok{2}\NormalTok{],}\DataTypeTok{col =} \DecValTok{2}\NormalTok{,}\DataTypeTok{lty =}\DecValTok{2}\NormalTok{)}
\KeywordTok{par}\NormalTok{(}\DataTypeTok{new =}\NormalTok{ T)}
\KeywordTok{plot}\NormalTok{(Xtest,Ytest,}\DataTypeTok{col =} \StringTok{'blue4'}\NormalTok{,}\DataTypeTok{xlab =} \StringTok{"Year"}\NormalTok{,}\DataTypeTok{ylab =} \StringTok{"Number of sale"}\NormalTok{,}\DataTypeTok{panel.first =} \KeywordTok{grid}\NormalTok{(}\DecValTok{10}\NormalTok{))\}}
\end{Highlighting}
\end{Shaded}

\includegraphics{exo4_files/figure-latex/unnamed-chunk-5-1.pdf}

\begin{Shaded}
\begin{Highlighting}[]
\NormalTok{\{}\KeywordTok{plot}\NormalTok{(Xtest,Yhat }\OperatorTok{-}\StringTok{ }\NormalTok{Ytest,}\DataTypeTok{xlab =} \StringTok{"Year"}\NormalTok{,}\DataTypeTok{ylab =} \StringTok{"misfit"}\NormalTok{, }\DataTypeTok{col =} \StringTok{"red"}\NormalTok{,}\DataTypeTok{panel.first =} \KeywordTok{grid}\NormalTok{(}\DecValTok{10}\NormalTok{))}
\KeywordTok{abline}\NormalTok{(}\DecValTok{0}\NormalTok{,}\DecValTok{0}\NormalTok{,}\DataTypeTok{col =} \DecValTok{1}\NormalTok{,}\DataTypeTok{lty =}\DecValTok{2}\NormalTok{)\}}
\end{Highlighting}
\end{Shaded}

\includegraphics{exo4_files/figure-latex/unnamed-chunk-5-2.pdf} we can
see how bad is the linear model, it doesn't fit to data. And the trend
show that a prediction will give data far from the reality.

Exemple let's predict Year 2020 and 2021

\begin{Shaded}
\begin{Highlighting}[]
\NormalTok{Xpred  =}\StringTok{ }\KeywordTok{matrix}\NormalTok{(}\KeywordTok{c}\NormalTok{(}\DecValTok{1}\NormalTok{,}\DecValTok{1}\NormalTok{,}\DecValTok{2020}\NormalTok{,}\DecValTok{2021}\NormalTok{),}\DecValTok{2}\NormalTok{,}\DecValTok{2}\NormalTok{)}
\NormalTok{Ypred =}\StringTok{ }\NormalTok{Xpred}\OperatorTok\NormalTok{betah}
\NormalTok{misfit =}\StringTok{ }\NormalTok{Ypred }\OperatorTok{-}\StringTok{ }\NormalTok{Yfut}
\end{Highlighting}
\end{Shaded}

We are going to look a for a new model.

ATTENTION !!!!!!

PAS FINI!!!!!!! We are oing to try the sigmoid function on these data if
\(y = f(x)\) with \(f(x) = \frac{1}{1+exp(-\lambda x)}\) We are going to
estimate the best \(\lambda\) The inverse of \(f\) is \$f\^{}\{-1\}(x) =
-(1/\lambda) * ln(\frac{1}{y}-1) = \alpha * ln(\frac{1}{y}-1) \$

We are going to estimate \(\alpha\) through a linear regression between
\(x = year\) and \(ln(\frac{1}{y}-1)\) avec \(y = N_{sale}\)

Data

\begin{Shaded}
\begin{Highlighting}[]
\NormalTok{Ytestb =}\StringTok{ }\OperatorTok{-}\KeywordTok{log}\NormalTok{(}\DecValTok{1}\OperatorTok{-}\NormalTok{(}\DecValTok{1}\OperatorTok{/}\NormalTok{Ytest))}
\KeywordTok{dim}\NormalTok{(tab)}
\end{Highlighting}
\end{Shaded}

\begin{verbatim}
## [1] 15  2
\end{verbatim}

\begin{Shaded}
\begin{Highlighting}[]
\NormalTok{Y1 =}\StringTok{ }\KeywordTok{cbind}\NormalTok{(}\KeywordTok{rep}\NormalTok{(}\DecValTok{1}\NormalTok{,}\KeywordTok{length}\NormalTok{(Ytestb)),Ytestb)}
\NormalTok{Y1}
\end{Highlighting}
\end{Shaded}

\begin{verbatim}
##               Ytestb
##  [1,] 1 0.0082088788
##  [2,] 1 0.0072051612
##  [3,] 1 0.0058180290
##  [4,] 1 0.0033766704
##  [5,] 1 0.0021208916
##  [6,] 1 0.0014714324
##  [7,] 1 0.0010317576
##  [8,] 1 0.0008037035
##  [9,] 1 0.0007025432
## [10,] 1 0.0006686906
## [11,] 1 0.0006511392
## [12,] 1 0.0006427686
## [13,] 1 0.0006560216
\end{verbatim}

\begin{Shaded}
\begin{Highlighting}[]
\NormalTok{tab2 =}\StringTok{ }\KeywordTok{as.data.frame}\NormalTok{(}\KeywordTok{cbind}\NormalTok{(Xtest,Ytestb))}
\NormalTok{tab2}
\end{Highlighting}
\end{Shaded}

\begin{verbatim}
##    Xtest       Ytestb
## 1   2007 0.0082088788
## 2   2008 0.0072051612
## 3   2009 0.0058180290
## 4   2010 0.0033766704
## 5   2011 0.0021208916
## 6   2012 0.0014714324
## 7   2013 0.0010317576
## 8   2014 0.0008037035
## 9   2015 0.0007025432
## 10  2016 0.0006686906
## 11  2017 0.0006511392
## 12  2018 0.0006427686
## 13  2019 0.0006560216
\end{verbatim}

\begin{Shaded}
\begin{Highlighting}[]
\NormalTok{regmod =}\StringTok{ }\KeywordTok{lm}\NormalTok{(Xtest }\OperatorTok{~}\StringTok{ }\NormalTok{., }\DataTypeTok{data =}\NormalTok{ tab2)}
\KeywordTok{summary}\NormalTok{(regmod)}
\end{Highlighting}
\end{Shaded}

\begin{verbatim}
## 
## Call:
## lm(formula = Xtest ~ ., data = tab2)
## 
## Residuals:
##     Min      1Q  Median      3Q     Max 
## -2.5489 -1.8921  0.0107  0.9592  3.6445 
## 
## Coefficients:
##               Estimate Std. Error  t value Pr(>|t|)    
## (Intercept)  2016.1645     0.7916 2546.862  < 2e-16 ***
## Ytestb      -1233.2708   215.6790   -5.718 0.000135 ***
## ---
## Signif. codes:  0 '***' 0.001 '**' 0.01 '*' 0.05 '.' 0.1 ' ' 1
## 
## Residual standard error: 2.041 on 11 degrees of freedom
## Multiple R-squared:  0.7483, Adjusted R-squared:  0.7254 
## F-statistic:  32.7 on 1 and 11 DF,  p-value: 0.0001345
\end{verbatim}

\begin{Shaded}
\begin{Highlighting}[]
\NormalTok{betah2 =}\StringTok{ }\NormalTok{regmod}\OperatorTok{$}\NormalTok{coefficients}
\NormalTok{lambda=}\StringTok{ }\NormalTok{betah2[}\DecValTok{2}\NormalTok{]}

\NormalTok{Xhat2 =}\StringTok{ }\NormalTok{Y1 }\OperatorTok\StringTok{ }\NormalTok{betah2}
\NormalTok{Xhat2 }
\end{Highlighting}
\end{Shaded}

\begin{verbatim}
##           [,1]
##  [1,] 2006.041
##  [2,] 2007.279
##  [3,] 2008.989
##  [4,] 2012.000
##  [5,] 2013.549
##  [6,] 2014.350
##  [7,] 2014.892
##  [8,] 2015.173
##  [9,] 2015.298
## [10,] 2015.340
## [11,] 2015.362
## [12,] 2015.372
## [13,] 2015.355
\end{verbatim}

\begin{Shaded}
\begin{Highlighting}[]
\KeywordTok{plot}\NormalTok{(Xhat2,Ytestb)}
\end{Highlighting}
\end{Shaded}

\includegraphics{exo4_files/figure-latex/unnamed-chunk-7-1.pdf}

We have the \(\lambda\) for the simoid function now we can build the
model

\begin{Shaded}
\begin{Highlighting}[]
\NormalTok{Yhat2 =}\StringTok{ }\DecValTok{1}\OperatorTok{/}\NormalTok{(}\DecValTok{1}\OperatorTok{+}\KeywordTok{exp}\NormalTok{(lambda}\OperatorTok{*}\NormalTok{Xhat2))}
\NormalTok{Xhat2}
\end{Highlighting}
\end{Shaded}

\begin{verbatim}
##           [,1]
##  [1,] 2006.041
##  [2,] 2007.279
##  [3,] 2008.989
##  [4,] 2012.000
##  [5,] 2013.549
##  [6,] 2014.350
##  [7,] 2014.892
##  [8,] 2015.173
##  [9,] 2015.298
## [10,] 2015.340
## [11,] 2015.362
## [12,] 2015.372
## [13,] 2015.355
\end{verbatim}

\begin{Shaded}
\begin{Highlighting}[]
\NormalTok{lambda}
\end{Highlighting}
\end{Shaded}

\begin{verbatim}
##    Ytestb 
## -1233.271
\end{verbatim}

\begin{Shaded}
\begin{Highlighting}[]
\NormalTok{Yhat2}
\end{Highlighting}
\end{Shaded}

\begin{verbatim}
##       [,1]
##  [1,]    1
##  [2,]    1
##  [3,]    1
##  [4,]    1
##  [5,]    1
##  [6,]    1
##  [7,]    1
##  [8,]    1
##  [9,]    1
## [10,]    1
## [11,]    1
## [12,]    1
## [13,]    1
\end{verbatim}

\begin{Shaded}
\begin{Highlighting}[]
\KeywordTok{plot}\NormalTok{(Xtest,Yhat2)}
\end{Highlighting}
\end{Shaded}

\includegraphics{exo4_files/figure-latex/unnamed-chunk-8-1.pdf}

Note that the \texttt{echo\ =\ FALSE} parameter was added to the code
chunk to prevent printing of the R code that generated the plot.

\end{document}
